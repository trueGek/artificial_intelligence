%%This is a very basic article template.
%%There is just one section and two subsections.
\documentclass[11pt,a4paper,italian]{extarticle}
\usepackage[italian]{babel}
\usepackage[utf8x]{inputenc}
\usepackage{graphicx}%per usare le immagini
\usepackage{amsmath}
\usepackage{amsfonts}
\usepackage{amssymb}
\usepackage{color}
\usepackage{hyperref}
\usepackage[all]{hypcap}
\usepackage{ifthen}
\usepackage{wrapfig}
\usepackage{indentfirst}
\usepackage{fancyhdr} %pacchetto per le intestazioni
\pagestyle{fancy} %uso del pacchetto
\usepackage{float}
\usepackage{pdflscape}
\usepackage{subfigure}

\fancyhead{} %annulla head di default
\fancyfoot{} %annulla foot di default
\usepackage{lastpage} %setto pg di pgtot a rfoot
\usepackage{booktabs}%pacchetto per tabelle nuove
\usepackage{eurosym} %per avere simbolo euro
\usepackage{multirow}%per avere celle su piu righe


\usepackage[top=2cm,bottom=4cm,left=80pt,right=80pt]{geometry} %disegna la linea
\setlength{\headheight}{2cm} %settato grandezza header

\rfoot{pagina \thepage\ di \pageref{LastPage}}
\lfoot{Versione: \versione} %setto versione doc a lfoot

\renewcommand{\footrulewidth}{1.5pt} %ridefinisco il valore della riga di intestazione
\renewcommand{\headrulewidth}{1.5pt} %ridefinisco il valore della riga di pie' di pagina
\addtolength{\headwidth}{\marginparsep}
\addtolength{\headwidth}{\marginparwidth}


\hypersetup
{
	colorlinks=true,
	linkcolor=blue,
	urlcolor=blue
}

\cfoot{
	\titolo \\ 
}


\newenvironment{infodocumento}{%
	\begin{center}%
		\begin{tabular}{r|l}%
            \multicolumn{2}{c}{\large{\textbf{Informazioni sul documento}}}\\%
			\hline%
			
}%
{
		\multirow{2}{*}{Distribuzione} & Alessandro Zonta, il tutor aziendale \\& e il tutor interno.
		\end{tabular}%
	\end{center}%
}

\newboolean{redarre}
\newboolean{approvare}
\newboolean{verificare}
\newboolean{distribuire}
\setboolean{redarre}{true}
\setboolean{approvare}{true}
\setboolean{verificare}{true}
\setboolean{distribuire}{true}
\newcommand{\Nome}[1]{\textbf{Nome}&#1\\}
\newcommand{\Creazione}{\textbf{Data di creazione}&\datacreazione\\}
\newcommand{\Versione}{\textbf{Versione}&\versione\\}
\newcommand{\UltimaModifica}{\textbf{Data di ultima modifica}&\today\\}
\newcommand{\Formale}{\textbf{Stato}&Formale\\}
\newcommand{\Preliminare}{\textbf{Stato}&Preliminare\\}
\newcommand{\Esterno}{\textbf{Uso}&Esterno\\}
\newcommand{\Interno}{\textbf{Uso}&Interno\\}


\newcommand{\Reda}[1]{\hline \textbf{Redazione}&\parbox{0.65\textwidth}{#1}\\}

\newcommand{\Ver}[1] {\hline \textbf{Verificatore}&\parbox{0.65\textwidth}{#1}\\}

\newcommand{\App}[1] {\hline \textbf{Approvazione}&\parbox{0.65\textwidth}{#1}\\}
  



\newenvironment{regmodifiche}{%
	\vspace{60px}
	\begin{Large}
    	\textbf{Registro delle modifiche}\\\\%
	 \end{Large}\\
	 \begin{tabular}{|c|c|c|}%
    	\hline
    	\parbox{0.25\textwidth}{Data \& Autore} & 
    	\parbox{0.1\textwidth}{Versione} & 
    	\parbox{0.57\textwidth}{Modifiche}\\
    	\hline
 	\end{tabular}
}%

\newcommand{\modifica}[4]{
		\begin{tabular}{|c|c|c|}%
			\parbox{0.25\textwidth}{\ \\ #1 \\ #3\\} & 
			\parbox{0.1\textwidth}{#2} & 
			\parbox{0.57\textwidth}{#4}\\ 
			\hline
		\end{tabular}%
}
